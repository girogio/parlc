\section{Project Structure}

The \code{ParL} compiler is implemented in Rust, and is structured as a Rust
crate providing a CLI interface. The folder structure of the project reflects
the fact that the compiler was built incrementally, with each stage having the
core implementation in its own module (folder).

\subsection{Dependencies}

The use of external crate dependencies has been kept to a minimum throughout
this assignment. Currently, the only external dependencies are:

\begin{itemize}

      \item \code{clap}: A crate that provides a convenient way to define
            command-line interfaces.

      \item \code{thiserror}: A crate that provides a convenient way to define
            custom error types.

      \item \code{console}: A crate that allows for coloured printing to \code{stdout}, for decoration purposes.
\end{itemize}

\subsection{Modules}

The $\code{ParL}$ compiler is structured as a Rust crate, with the following
modules:

\begin{itemize}
      \item \code{core}: Contains the core data structures, such as the
            \code{Token} struct, the \code{TokenKind} enum and the \code{AstNode} enum,
            which are used throughout the compiler, in the majority of the stages.
      \item \code{lexing}: Contains the lexer implementation, together with its
            requirements such as the \code{Dfsa} and the \code{Lexer} struct.
      \item
            \code{parsing}: Contains the parser implementation, together with the
            \code{Parser} struct and the \code{TreePrinter} visitor.
      \item \code{semantics}: Contains a \code{visitors} folder, which
            in turn contains a number of visitors, namely the \code{SemanticAnalyser}, the \code{Formatter}, and the \code{TreePrinter}.
      \item \code{generation}: Contains the \code{ParIR} code generator visitor,
            as well as an abstraction of the \code{ParIR} language, instructions.
\end{itemize}

\newpage

\subsection{Quick Start}

The \code{ParL} compiler has a command-line tool interface, and these
are the commands that can be used to interact with. The initial run for any of
these commands will cause the Rust project to be built, so the first run might
take a bit longer than usual.

\textbf{Getting help:} $$\code{cargo run --\,-- --\,--help}$$

\textbf{Formatting a \code{ParL} file:} $$\code{cargo run fmt path/to/file.parl}$$

The above command will format the input file according to the a predetermined
style guide, and directly modify the file in place.

\textbf{Running the lexer:} $$\code{cargo run  lex path/to/file.parl}$$

The above command will print to \code{stdout} the tokens that the lexer has extracted from the input file, one per line.

\textbf{Running the parser:} $$\code{cargo run parse path/to/file.parl}$$

The above command will print to \code{stdout} the abstract syntax tree (AST)
that the parser has generated from the input file, in a pretty-printed format
provided by the \code{TreePrinter} visitor.

\textbf{Running the semantic analyser:} $$\code{cargo run sem path/to/file.parl}$$

The above command will print to \code{stdout} any semantic errors and/or warnings that the semantic analyser has found in the input file.

\textbf{Running the full compiler:} $$\code{cargo run  compile path/to/file.parl}$$

The above command will compile the input file to \code{PArIR} and print the
generated code to \code{stdout}. If the \code{-o output\_file.parir} flag is
provided, the output will be written to the specified file.

Note that all compilation related commands cause the previous steps to be run as
well, although only the output of the requested step is printed to
\code{stdout}.
