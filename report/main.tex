\documentclass{article}

\usepackage{subcaption}
\usepackage[utf8]{inputenc}
\usepackage[english]{babel}
\usepackage[backend=bibtex]{biblatex}
\usepackage{tikz}
\usepackage{xcolor}
\usepackage{multicol}
\usepackage{listings}
\usepackage[most]{tcolorbox}
\usepackage{parskip}
\usepackage{listings-rust}
\usepackage[scale=0.9]{sourcecodepro}
\usepackage{bytefield}
\usepackage{tikz}
\usepackage{adjustbox}

\addbibresource{ref}


\usetikzlibrary{arrows,calc,positioning}
\tikzstyle{intt}=[draw,text centered,minimum size=6em,text width=5.25cm,text height=0.34cm]
\tikzstyle{intl}=[draw,text centered,minimum size=2em,text width=2.75cm,text height=0.34cm]
\tikzstyle{int}=[draw,minimum size=2.5em,text centered,text width=3.5cm]
\tikzstyle{intg}=[draw,minimum size=3em,text centered,text width=6.cm]
\tikzstyle{sum}=[draw,shape=circle,inner sep=2pt,text centered,node distance=3.5cm]
\tikzstyle{summ}=[drawshape=circle,inner sep=4pt,text centered,node distance=3.cm]

% a4 marigns
\usepackage[a4paper, total={6in, 8in}]{geometry}

\tcbset {
  base/.style={
    arc=0mm,
    bottomtitle=0.5mm,
    boxsep=1mm,
    boxrule=0mm,
    colbacktitle=black!10!white,
    coltitle=black,
    fonttitle=\bfseries,
    left=2.5mm,
    leftrule=1mm,
    right=3.5mm,
    title={#1},
    toptitle=0.75mm,
  },
    sub/.style={
        base={#1},
        colframe=black!30!white,
        top=-0.5mm,
        bottom=-0.5mm,
    },
}

\definecolor{brandblue}{rgb}{0.34, 0.7, 1}
\newtcolorbox{mainbox}[1]{
    nobeforeafter,
    colframe=brandblue,
    base={#1}
}

\newtcolorbox{subbox}[1]{
  colframe=black!30!white,
  sub={#1}
}

% warning mainbox
\newtcolorbox{warningbox}[1]{
  colframe=red,
  base={#1}
}

\usepackage{lato}
\renewcommand*\familydefault{\sfdefault}
\newcommand{\code}[1]{\texttt{#1}}
\usepackage[T1]{fontenc}
\usepackage{hyperref}
\usepackage{fancyhdr}

\pagestyle{fancy}
\fancyhf{}
\rhead{Giorgio Grigolo}
\lhead{CPS 2000: Compiler Theory and Practice}
\rfoot{Page \thepage}

\definecolor{bluekeywords}{rgb}{0.13, 0.13, 1}
\definecolor{greencomments}{rgb}{0, 0.5, 0}
\definecolor{redstrings}{rgb}{0.9, 0, 0}
\definecolor{graynumbers}{rgb}{0.5, 0.5, 0.5}


\lstset{
    commentstyle=\color{greencomments},
    keywordstyle=\color{bluekeywords},
    stringstyle=\color{redstrings},
    numberstyle=\color{graynumbers},
    % breaklines=true,
    basicstyle=\ttfamily,
    language=Rust,
    xleftmargin=.3\textwidth, xrightmargin=.3\textwidth,
    captionpos=b
}

\title{Compiling \code{ParL} to \code{PArIR} in Rust \\{\normalsize A report on the Rust
compiler for the \code{ParL} programming language.}}
\author{Giorgio Grigolo - 0418803L}
\date{}

\hypersetup{
    colorlinks=true,
    linkcolor=blue,
    filecolor=magenta,
    urlcolor=blue,
}

\begin{document}

\maketitle
\tableofcontents

\begin{abstract}
    In this report, we discuss the implementation details of a
    compiler for \code{ParL}, an expression-based strongly typed programming
    langauge. Code written in \code{ParL} is compiled to \code{PArIR}, which is
    the proprietary assembly-like language that is used to drive the
    programmable pixel art displays designed by the company \code{PArDis}. The
    \code{ParL} compiler was written in Rust, due to its strong type system and
    performance characteristics. It nwas implemented incrementally, as
    to ensure each component can be run in isolation, and is working correctly
    before moving on to the next one.
\end{abstract}

\newpage

\section{Project Structure}

\section{Lexical Analysis}

The first step in the compilation process is the lexical analysis. A recurring
theme in the implementation of this compiler is the use of \textit{abstraction}
to achieve \textit{modularity}, and simplify the implementation of the
subsequent stages. The lexical analysis is no exception to this rule.

At the highest level, the lexical analysis is implemented as a \code{Lexer}
struct, which is responsible for reading the input source code, and producing a
stream (\textit{or vector}) of tokens.

We shall start by defining the \code{Token} struct, which represents the different
types of tokens that can be found in a \code{ParL} program. The \code{Token}
enum is defined as follows:

\begin{mainbox}{}
    \begin{lstlisting}[language=Rust]
pub struct Token {
    pub kind: TokenKind,
    pub span: TextSpan,
}
    \end{lstlisting}
\end{mainbox}



\cite{engineering-a-compiler}


\newpage

\printbibliography


\end{document}
