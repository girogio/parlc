\section{Code Generation}

The final step in the compilation process is the code generation phase. In this phase, the compiler takes the AST and generates the \code{PArIR} code that will be executed by the \code{PArDis} displays.

\subsection{Abstracting Instructions}

The first step in the code generation phase was to abstract the \code{PArIR}
instructions into a Rust enum. This enum, called \code{Instruction}, contains
all the instructions that the \code{PArDis} displays can execute. The
instruction variants containing arguments, are represented as part of the enum
in an unnamed tuple struct. These instructions are all of the \code{Push}
variants, and the function labels, which contain the memory location/value to
push and the name of the function respectively.

The \code{Instruction} enum has a \code{Display} trait implementation, which is
equivalent to a \code{toString} method in other languages. We will set the
string representation of the instruction to be the same as the \code{PArIR}
instruction itself, with the arguments formatted accordingly, as it will be
ouputted in the final \code{PArIR} code.
