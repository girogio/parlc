\section{Testing Framework}

\subsection{Unit Testing}

During the early stages of development, such as the lexer and parser, most of
the testing was done through unit tests, first with smaller cases, and then with
larger cases.

\subsection{Integration Testing}

After the lexer and parser were completed, integration testing was done to
ensure that the lexer and parser worked together correctly. This was done by
running the CLI interface, one stage at a time, on small, simple programs.

Nearing the end of the project, I decided that it would be useful to create a
more robust testing framework to test the entire compiler, as well as the
compiled code on the VM. Using the Python library
\href{https://playwright.dev/python/}{Playwright} I created a script that opens
the VM website in a browser (in headless mode) and runs the compiled code. The
script then gets the output of the log, transforms it into an array of its
lines, and removes the three \code{`>'} symbols at the beginning of each log
entry.

In Python then, we are free to use this array to check against the expected
values, through assertions, and we print out the results of the tests.

As a sanity check, the provided programs in the assignment rubric, namely the
race program, and the sample program for arrays, have been included in the
samples, and were tested by hand, making sure they all work as expected.



\subsection{Results}

\subsubsection*{Variable declarations}

\textbf{ParL Program:}

\begin{lstlisting}
    let a: int = 5;
    let b: colour = #000000;
    let c: float  = 1.23;

    let d: int[5] = [1, 2, 3, 4, 5];

    __print a;
    __print b;
    __print c;
    __print d;

    {
        let a: int = 1;
        __print a;
    }

    __print a;
\end{lstlisting}

\newpage

\textbf{AST:}

\begin{lstlisting}
Program
  VarDec
    Identifier: a
    Type: int
    Expression: 5
  VarDec
    Identifier: b
    Type: colour
    Expression: #000000
  VarDec
    Identifier: c
    Type: float
    Expression: 1.23
  VarDecArray
    Identifier: d
    Element Type: int
    Size: 5
    Elements: 1, 2, 3, 4, 5
  Print
    Expression: a
  Print
    Expression: b
  Print
    Expression: c
  Print
    Expression: d
  Block
    VarDec
      Identifier: a
      Type: int
      Expression: 1
    Print
      Expression: a
  Print
    Expression: a
\end{lstlisting}

\textbf{Result:}
Output successfully matched with $$\code{['5', '0', '1.23', '1', '2', '3', '4',
        '5', '1', '5']}$$

\newpage

\subsubsection*{Variable Assignment}

\textbf{ParL Program:}

\vfill

\begin{lstlisting}
    let a: int = 5;
    let b: colour = #000000;
    let c: float  = 1.23;
    let d: int[5] = [1, 2, 3, 4, 5];

    __print a;
    __print b;
    __print c;
    __print d[1];

    a = 2;
    b = #FFFFFF;
    c = 3.45;
    d[1] = 1;

    __print a;
    __print b;
    __print c;
    __print d[0];
\end{lstlisting}
\vfill

\newpage

\textbf{AST:}


\begin{lstlisting}[language=]
Program
  VarDec
    Identifier: a
    Type: int
    Expression: 5
  VarDec
    Identifier: b
    Type: colour
    Expression: #000000
  VarDec
    Identifier: c
    Type: float
    Expression: 1.23
  VarDecArray
    Identifier: d
    Element Type: int
    Size: 5
    Elements: 1, 2, 3, 4, 5
  Print
    Expression: a
  Print
    Expression: b
  Print
    Expression: c
  Print
    Expression: ArrayAccess
        Identifier: d
        Index: 1
  Assignment
    Identifier: a
    Expression: 2
  Assignment
    Identifier: b
    Expression: #FFFFFF
  Assignment
    Identifier: c
    Expression: 3.45
  Assignment
    Identifier: d
    Expression: 1
    Index: 1
  Print
    Expression: a
  Print
    Expression: b
  Print
    Expression: c
  Print
    Expression: ArrayAccess
        Identifier: d
        Index: 0
\end{lstlisting}


\textbf{Result:}

Output successfully matched with $$\code{["5", "0", "1.23", "2", "2", "16777215",
        "3.45", "1"]}$$


\subsubsection*{Control Flows}

\textbf{ParL Program:}

{

  \lstset{xleftmargin=0.2\textwidth}

  \begin{lstlisting}
    let loop_max: int = 5;

    for (let i: int = 0; i < loop_max; i = i + 1) {
        __print i;
    }

    if (1 < 2) {
        __print 1;
    } else {
        __print 0;
    }

    if (1 > 2) {
        __print 1;
    } else {
        __print 0;
    }


    let i: int = 0;
    while (i < loop_max) {
        __print i;
        i = i + 1;
    }
\end{lstlisting}

}

\newpage

\textbf{AST:}


\begin{lstlisting}
Program
  VarDec
    Identifier: a
    Type: int
    Expression: 5
  VarDec
    Identifier: b
    Type: colour
    Expression: #000000
  VarDec
    Identifier: c
    Type: float
    Expression: 1.23
  VarDecArray
    Identifier: d
    Element Type: int
    Size: 5
    Elements:
      1, 2, 3, 4, 5,
  Print
    Expression: a
  Print
    Expression: b
  Print
    Expression: c
  Print
    Expression: ArrayAccess
        Identifier: d
        Index: 1
  Assignment
    Identifier: a
    Expression: 2
  Assignment
    Identifier: b
    Expression: #FFFFFF
  Assignment
    Identifier: c
    Expression: 3.45
  Assignment
    Identifier: d
    Expression: 1
    Index: 1
  Print
    Expression: a
  Print
    Expression: b
  Print
    Expression: c
  Print
    Expression: ArrayAccess
        Identifier: d
        Index: 0
\end{lstlisting}

\textbf{Result:}

Output successfully matched with $$\code{["0", "1", "2", "3", "4", "1", "0", "0", "1", "2", "3", "4"]}$$

\newpage

\subsubsection*{Functions}

\textbf{ParL Program:}
{
  \lstset{xleftmargin=0.2\textwidth}

  \begin{lstlisting}
fun add(a: int, b: int) -> int {
    return a + b;
}

let result: int = add(5, 10);

__print result;

fun some_func(a: int[2]) -> int[2] {
    let b: int[2] = [1,3];
    return b;
}

let a: int[2] = [1, 2];

__print some_func(a);


fun mixed_parameters(a: int, b: float, c: colour) -> bool {
    __print a;
    __print b;
    __print c;

    return true;
}

__print mixed_parameters(5, 1.23, #000000);

fun array_parameters(a: int[5]) -> int {

    let sum: int = 0;

    for (let i: int = 0; i < 5; i = i + 1) {
        __print a[i];
        sum = sum + a[i];
    }

    return sum;
}

let arr: int[5] = [5, 1, 2, 3, 4];

__print array_parameters(arr);

\end{lstlisting}
}

\newpage

\textbf{AST:}

{

  \lstset{xleftmargin=0.25\textwidth}

  \begin{lstlisting}
Program
  FunctionDecl
    Identifier: add
    Params: a: int, b: int,
    Return Type: int
    Block: Block
      Return
        Expression: (a + b)
  VarDec
    Identifier: result
    Type: int
    Expression: add(5, 10)
  Print
    Expression: result
  FunctionDecl
    Identifier: some_func
    Params: a: int[2],
    Return Type: int[2]
    Block: Block
      VarDecArray
        Identifier: b
        Element Type: int
        Size: 2
        Elements:
          1, 3,
      Return
        Expression: b
  VarDecArray
    Identifier: a
    Element Type: int
    Size: 2
    Elements:
      1, 2,
  Print
    Expression: some_func(a)
  FunctionDecl
    Identifier: mixed_parameters
    Params: a: int, b: float, c: colour,
    Return Type: bool
    Block: Block
      Print
        Expression: a
      Print
        Expression: b
      Print
        Expression: c
      Return
        Expression: true
  Print
    Expression: mixed_parameters(5, 1.23, #000000)
  FunctionDecl
    Identifier: array_parameters
    Params: a: int[5],
    Return Type: int
    Block: Block
      VarDec
        Identifier: sum
        Type: int
        Expression: 0
      For
        Initializer: VarDec
          Identifier: i
          Type: int
          Expression: 0
        Condition: (i < 5)
        Increment:
          Assignment
            Identifier: i
            Expression: (i + 1)

        Body: Block
          Print
            Expression: ArrayAccess
                Identifier: a
                Index: i
          Assignment
            Identifier: sum
            Expression: (sum + ArrayAccess
                Identifier: a
                Index: i)
      Return
        Expression: sum
  VarDecArray
    Identifier: arr
    Element Type: int
    Size: 5
    Elements:
      5, 1, 2, 3, 4,
  Print
    Expression: array_parameters(arr)
\end{lstlisting}

}

\textbf{Result:}

Output successfully matched with $$\code{["15", "1", "3", "5", "1.23", "0", "1",
        "5", "1", "2", "3", "4", "15"]}$$

\newpage

\subsubsection*{Expressions}

\textbf{ParL Program:}

{
  \lstset{xleftmargin=0.1\textwidth}

  \begin{lstlisting}
let a: int = 5;
let b: int = 10;

__print (a + b) == 20;
__print ((((a + b) * 2) / 3) - 1) == 9;
__print (a + b) <= 20 and a - b <= 0;
__print ((12 + 5) * (8 - 3) + (18 / 2)) - ((25 - 5) / (4 + 1));
\end{lstlisting}

}

\textbf{AST:}

{

  \lstset{xleftmargin=0\textwidth}

  \begin{lstlisting}
Program
  VarDec
    Identifier: a
    Type: int
    Expression: 5
  VarDec
    Identifier: b
    Type: int
    Expression: 10
  Print
    Expression: (((a + b)) == 20)
  Print
    Expression: (((((((((a + b)) * 2)) / 3)) - 1)) == 9)
  Print
    Expression: ((((a + b)) <= 20) and ((a - b) <= 0))
  Print
    Expression: ((((((12 + 5)) * ((8 - 3))) + ((18 / 2)))) - ((((25 - 5)) / ((4 + 1)))))
\end{lstlisting}

}

\textbf{Result:}

Output successfully matched with $$\code{["0", "1", "1", "90"]}$$

\newpage

\subsubsection*{Pad Read/Write}

\textbf{ParL Program:}

{
  \lstset{xleftmargin=0.2\textwidth}

  \begin{lstlisting}
let c1: colour = #213B5F;
let c2: colour = #FFA400;

let x1: int = 5;
let y1: int = 10;

let x2: int = 20;
let y2: int = 30;

__write x1, y1, c1;

__print __read x1, y1;

__write_box x1, y1, x2, y2, c2;

__print __read x1 + 3, y1 + 3;
\end{lstlisting}

}

\textbf{AST:}

{
  \lstset{xleftmargin=0.2\textwidth}

  \begin{lstlisting}
Program
    VarDec
      Identifier: c1
      Type: colour
      Expression: #213B5F
    VarDec
      Identifier: c2
      Type: colour
      Expression: #FFA400
    VarDec
      Identifier: x1
      Type: int
      Expression: 5
    VarDec
      Identifier: y1
      Type: int
      Expression: 10
    VarDec
      Identifier: x2
      Type: int
      Expression: 20
    VarDec
      Identifier: y2
      Type: int
      Expression: 30
    __write x1, y1, c1;
    Print
      Expression: __read x1, y1
    __write_box x1, y1, x2, y2, c2;
    Print
      Expression: __read (x1 + 3), (y1 + 3)
\end{lstlisting}

}

\textbf{Result:}

Output successfully matched with $$\code{["2177887", "16753664"]}$$

\newpage
