\section{Code Generation}

The final step in the compilation process is the code generation phase. In this phase, the compiler takes the AST and generates the \code{PArIR} code that will be executed by the \code{PArDis} displays.

\subsection{Abstracting Instructions}

The first step in the code generation phase was to abstract the \code{PArIR}
instructions into a Rust enum. This enum, called \code{Instruction}, contains
all the instructions that the \code{PArDis} displays can execute. Some of the
instructions, like \code{Push}, \code{Pusha} and really most of the push
variants, take some arguments, which are stored in the enum itself, as an
unnamed tuple e.g. \code{PushValue(usize)}.

The \code{Instruction} enum has a \code{Display} trait implementation, which is
equivalent to a \code{toString} method in other languages. We will set the
string representation of the instruction to be the same as the \code{PArIR}
instruction itself, with the arguments formatted accordingly, as it will be
outputted in the final \code{PArIR} code.

\subsection{Program Intermediate Representation}

Fundamentally, we can represent a \code{PArIR} program as an ordered list of
instructions. Furthermore, these instructions can be grouped in two sections:
\begin{itemize}
    \item The \textit{function section}, which contains all the function definitions.
    \item The \textit{main section}, the entrypoint of the program, which
          contains the instructions that will be executed when the program starts.
\end{itemize}

In fact, our \code{Program} struct, is exactly the above.

\begin{mainbox}{}
    \lstset{xleftmargin=.2\textwidth, aboveskip=0pt, belowskip=0pt}
    \begin{lstlisting}
struct Program {
    pub functions: Vec<Instruction>,
    pub main: Vec<Instruction>,
}
\end{lstlisting}
\end{mainbox}

When writing the resulting \code{PArIR} code to a file, we first write the
function section, followed by the main section. Due to the fact that the only
thing we need to call a function, is its name, not the line number where it has
been defined, we can successfully divide the program into these two sections.

\newpage

\subsection{Code Generation Visitor}

The code generation visitor is the final visitor that is run on the AST.
Contrary to the the semantic analyser, instead of returning type information as
it visits the nodes of the AST, it returns \textit{line numbers} of the
generated \code{PArIR} code. These line numbers are mostly used to generate all
kinds \code{Jump} instructions correctly. For example, when we generate an if
statement, we need to know the line number of the first instruction of the true
branch, and that of the false branch, so that upon evaluating the condition, we
know where to jump.

Of course, other things need to be kept track of, like the current stack level,
and the current frame index, which are used to generate the correct \code{Push}
instructions, when referencing variables from the symbol table. Here is the full
definition of the \code{PArIRWriter} struct, upon which we implement the visitor
trait.

\begin{mainbox}{}
    \lstset{xleftmargin=.1\textwidth, aboveskip=0pt, belowskip=0pt}
    \begin{lstlisting}
pub struct PArIRWriter {
        /// Stack of symbol tables, each representing a scope
        symbol_table: Vec<SymbolTable>,
        /// The ParIR program container
        program: Program,
        /// Pointer to the current instruction
        instr_ptr: usize,
        /// The current stack level
        stack_level: usize,
        /// The current frame index
        frame_index: usize,
    }
\end{lstlisting}
\end{mainbox}

Additionally, we also have some helper functions such as

\begin{itemize}
    \item \code{get\_scope\_var\_count}, which returns the number of slots in
          the that we need to allocate in the stack frame for the current scope.
    \item \code{get\_memory\_location(token: Token)}, which returns the memory
          location of a variable, given its name.  \item \code{push\_scope()} and
          \code{pop\_scope()}, being common operations, we shall automate them, whilst
          resetting the frame index and incrementing the stack level.
\end{itemize}

